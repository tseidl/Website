\documentclass[12pt,]{article}
\usepackage[margin=1in]{geometry}
\newcommand*{\authorfont}{\fontfamily{phv}\selectfont}
\usepackage[]{helvet}
\usepackage{abstract}
\renewcommand{\abstractname}{}    % clear the title
\renewcommand{\absnamepos}{empty} % originally center
\newcommand{\blankline}{\quad\pagebreak[2]}

\providecommand{\tightlist}{%
  \setlength{\itemsep}{0pt}\setlength{\parskip}{0pt}} 
\usepackage{longtable,booktabs}

\usepackage{parskip}
\usepackage{titlesec}
\titlespacing\section{0pt}{12pt plus 4pt minus 2pt}{6pt plus 2pt minus 2pt}
\titlespacing\subsection{0pt}{12pt plus 4pt minus 2pt}{6pt plus 2pt minus 2pt}

\titleformat*{\subsubsection}{\normalsize\itshape}

\usepackage{titling}
\setlength{\droptitle}{-.25cm}

%\setlength{\parindent}{0pt}
%\setlength{\parskip}{6pt plus 2pt minus 1pt}
%\setlength{\emergencystretch}{3em}  % prevent overfull lines 

\usepackage[T1]{fontenc}
\usepackage[utf8]{inputenc}

\usepackage{fancyhdr}
\pagestyle{fancy}
\usepackage{lastpage}
\renewcommand{\headrulewidth}{0.3pt}
\renewcommand{\footrulewidth}{0.0pt} 
\lhead{\footnotesize Anke Obendiek, Timo Seidl}
\chead{}
\rhead{\footnotesize Digital Power Europe? The EU in a Digitalizing
World -- Summer 2021}
\lfoot{}
\cfoot{\small \thepage/\pageref*{LastPage}}
\rfoot{}

\fancypagestyle{firststyle}
{
\renewcommand{\headrulewidth}{0pt}%
   \fancyhf{}
   \fancyfoot[C]{\small \thepage/\pageref*{LastPage}}
}

%\def\labelitemi{--}
%\usepackage{enumitem}
%\setitemize[0]{leftmargin=25pt}
%\setenumerate[0]{leftmargin=25pt}




\makeatletter
\@ifpackageloaded{hyperref}{}{%
\ifxetex
  \usepackage[setpagesize=false, % page size defined by xetex
              unicode=false, % unicode breaks when used with xetex
              xetex]{hyperref}
\else
  \usepackage[unicode=true]{hyperref}
\fi
}
\@ifpackageloaded{color}{
    \PassOptionsToPackage{usenames,dvipsnames}{color}
}{%
    \usepackage[usenames,dvipsnames]{color}
}
\makeatother
\hypersetup{breaklinks=true,
            bookmarks=true,
            pdfauthor={ ()},
             pdfkeywords = {},  
            pdftitle={Digital Power Europe? The EU in a Digitalizing
World},
            colorlinks=true,
            citecolor=blue,
            urlcolor=blue,
            linkcolor=magenta,
            pdfborder={0 0 0}}
\urlstyle{same}  % don't use monospace font for urls


\setcounter{secnumdepth}{0}





\usepackage{setspace}

\title{Digital Power Europe? The EU in a Digitalizing World}
\author{Anke Obendiek, Timo Seidl}
\date{Summer 2021}


\begin{document}  

		\maketitle
		
	
		\thispagestyle{firststyle}

%	\thispagestyle{empty}


\begin{tabular*}{\textwidth}{p{.5\textwidth} p{.5\textwidth}}

E-mail: \texttt{\href{mailto:anke.obendiek@univie.ac.at}{\nolinkurl{anke.obendiek@univie.ac.at}}} & Class Room: \emph{online}/Hörsaal
41 Hauptgebäude, 1.Stock, Stiege 8\\
Office Hours: By appointment  &  Class Hours: Thursday, 09:45-11:15\\
Office: \emph{online}  &  Web: TBD\\
	&  \\
	\hline
	\end{tabular*}
	
\vspace{2mm}
	


\hypertarget{course-description}{%
\section{Course Description}\label{course-description}}

In a rapidly digitalizing world, Europe has yet to find its place. Some
observers, including many European officials, lament that Europe has
failed to keep pace with the United States and China - the two digital
superpowers. They point to the fact Europe has yet to produce its own
digital champion (Spotify is neither an Amazon nor an Alibaba), and
raise the specter of European irrelevance in a world where political and
economic influence increasingly rest upon technological progress.
Meanwhile, other observers, again including many European officials,
paint a rosier picture. They point to the fact that the EU has become
the de-jure and de-facto regulator of large swaths of the internet, and
has established itself as the global vanguard of digital human rights
(the GDPR being the most famous example in both cases). But what is the
role that Europe can, does, and should play in this brave new world of
digital platforms, big data and artificial intelligence? In this
seminar, we probe this question more systematically, assessing the
position of the EU has and the limits of this power, asking ourselves
why Europe lags behind in some areas and why it leads in others, and
evaluating the possibility and desirability of a third model of
digitalization - parallel to or in competition with the United States's
market-driven laissez faire approach as well as China's state-led
autocratic model.

\hypertarget{learning-outcomes}{%
\section{Learning Outcomes}\label{learning-outcomes}}

The course aims to introduce students to central debates in the emerging
literature on digitalization, and focuses particularly on the role of
the EU It provides students with the conceptual tools and analytic
skills to embark on theoretical or empirical projects of their own. At
the end of the course, students should be able to

\begin{itemize}
\item
  identify and describe the main challenges that digitalization poses to
  advanced capitalist countries in general and to the European Union in
  particular;
\item
  summarize and critically assess the strengths and weakness of
  theoretical approaches that try make sense of Europe's differential
  success in the areas of technological/economic and
  regulatory/normative leadership;
\item
  apply the insights they gained from these approaches to their own
  projects.
\end{itemize}

\hypertarget{requirements}{%
\section{Requirements}\label{requirements}}

Students are required to attend classes and come prepared (i.e., having
finished and thought about the readings). In addition, there will be
three types of assignments that together make up the final grade.

\begin{itemize}
\item
  First, for \textbf{three} sessions of their choosing, students need to
  write short response papers (half a page) that reflect on the readings
  and end with a question for the class (25\%). And remember, questions
  end with a question mark.
\item
  Second, they are required to deliver a very short input presentation
  (around 5-10 minutes) for \textbf{one} session as well as prepare
  discussion points for the class (e.g., questions, empirical examples)
  (25\%). The former is meant to quickly summarize the main points of
  the reading(s) and the latter is meant to kick off and organize the
  discussion.\footnote{We will see how exactly we do this in the first
    session, after we know how many participants there are. But the idea
    is that no one person should talk for more than 5 minutes.}
\item
  Lastly, students need to write a relatively short term paper on a
  topic related to the course (up to 4000 words) (50\%). The paper can
  be theoretical or empirical and is meant to hone in on one particular
  question that the students can pick themselves (although they should
  briefly discuss this with us in advance).
\end{itemize}

\hypertarget{prerequisites}{%
\section{Prerequisites}\label{prerequisites}}

Students need no prior knowledge of academic debates on digitalization
to successfully participate in the course. A general interest in the
topic, basic English language skills, and a broad familiarity with the
European Union are sufficient.

\hypertarget{course-policy}{%
\section{Course Policy}\label{course-policy}}

Basically, don't cheat and try to learn stuff, details follow below.

\hypertarget{grading-policy}{%
\subsection{Grading Policy}\label{grading-policy}}

You need to submit all the required assignments to pass the course. Your
final grade will be the weighted average of these assignments. What is
important to us when it comes to grading are two things. First, stick to
the task at hand. If your response paper is meant to end with a
question, end with a question. If your presentation is meant to be 5
minutes, make it no more than 6. If your term paper is meant to discuss
one question in around 4000 words, don't try to answer half a dozen in
8000 words. It's almost a dad thing to say, but these skills are
important not just at a university, but in any kind of job. Second, put
a bit of effort into it, or at least make it look that way. Have some
decent formatting. But also try to be clear and crisp, which is often
harder than writing long and convoluted sentences. Try to prepare a
presentation that you yourself would like to listen to: short, clear
points, that highlight anything that you found confusing or unclear. You
don't need to understand everything, have read a ton of additional
literature, or write in a fancy way to get a very good grade. Just stick
to the task and try to make sense.

\hypertarget{e-mail-policy}{%
\subsection{E-mail Policy}\label{e-mail-policy}}

You can always email us if you have an idea for a term paper, if you
want to learn more about a certain topic and don't know where to start,
or if you have a question that you really don't want to ask in class.
Please don't email us with questions that you could easily find the
answer to in the syllabus or in previous emails. We might take it badly.

This is a tough time for many, so if you feel like you struggle to cope
or just need to get an outside perspective, please get in touch with the
Psychological Counseling Center:
\url{https://www.studierendenberatung.at/en}

\hypertarget{attendance-policy}{%
\subsection{Attendance Policy}\label{attendance-policy}}

You are required to attend each session, and we encourage you to prepare
for and actively participate in them. However, if you really can't make
it, just reach out to us, these things happen once or twice a term.

\newpage

\begingroup\huge

\textbf{Course Outline} \endgroup

\vspace{0.25cm}

\hypertarget{week-1-march-11-introduction}{%
\subsection{Week 1, March 11 :
Introduction}\label{week-1-march-11-introduction}}

\emph{No readings for this session}

\hypertarget{section}{%
\section{\texorpdfstring{\textcolor{Blue}{\textsc{Part I: Foundations of Digitalization}}}{}}\label{section}}

\vspace{0.5cm}

\hypertarget{week-2-march-18-the-rise-of-the-platform-society}{%
\subsection{Week 2, March 18 : The Rise of the Platform
Society}\label{week-2-march-18-the-rise-of-the-platform-society}}

van Dijck, J. (2020) `Governing digital societies: Private platforms,
public values', Computer Law \& Security Review, 36, pp.~1--4.

Srnicek, N. (2017) Platform Capitalism. Cambridge: Polity, pp.~36--92.

\hypertarget{week-3-march-25-artifical-intelligence-big-data}{%
\subsection{Week 3, March 25 : Artifical Intelligence \& Big
Data}\label{week-3-march-25-artifical-intelligence-big-data}}

MIT Technology Review and Oracle (2016) The Rise of Data Capital.
Available at:
\url{http://files.technologyreview.com/whitepapers/MIT_Oracle+Report-The_Rise_of_Data_Capital.pdf}.

MIT Work of the Future Task Force (2020) The Work of the Future:
Building Better Jobs in an Age of Intelligent Machines, pp.~1--6,
pp.~30--46. Available at:
\url{https://workofthefuture.mit.edu/research-post/the-work-of-the-future-building-better-jobs-in-an-age-of-intelligent-machines/}.

\hypertarget{week-4-april-1-spring-holiday}{%
\subsection{Week 4, April 1 : Spring
holiday}\label{week-4-april-1-spring-holiday}}

\emph{We do not have class this week due to public holidays}

\hypertarget{week-5-april-8-spring-holiday}{%
\subsection{Week 5, April 8 : Spring
holiday}\label{week-5-april-8-spring-holiday}}

\emph{We do not have class this week due to public holidays}

\hypertarget{section-1}{%
\section{\texorpdfstring{\textcolor{Blue}{\textsc{Part II: The European Union in the World}}}{}}\label{section-1}}

\vspace{0.5cm}

\hypertarget{week-6-april-15-normative-power-europe}{%
\subsection{Week 6, April 15 : Normative Power
Europe}\label{week-6-april-15-normative-power-europe}}

Manners, I. (2002) `Normative Power Europe: A Contradiction in Terms?'
JCMS: Journal of Common Market Studies, 40(2), pp.~235--258.

Diez, T. (2005). Constructing the Self and Changing Others:
Reconsidering `Normative Power Europe'. Millennium, 33(3), pp.~613--636.

\hypertarget{week-7-april-22-market-power-europe}{%
\subsection{Week 7, April 22 : Market Power
Europe}\label{week-7-april-22-market-power-europe}}

Damro, C. (2012) `Market Power Europe', Journal of European Public
Policy, 19(5), pp.~682--699.

Seoane, M. V. (2020) Normative market Europe? The contested governance
of cyber-surveillance technologies. In: Emerging Security Technologies
and EU Governance, Routledge, pp.~88--101.

\hypertarget{week-8-april-29-the-brussels-effect}{%
\subsection{Week 8, April 29 : The Brussels
Effect}\label{week-8-april-29-the-brussels-effect}}

Bradford, A. (2020) The Brussels effect: How the European Union rules
the world. New York, NY: Oxford University Press, Chapter 5. Digital
Economy, pp.~xiii-xix, pp.~131-170.

\hypertarget{section-2}{%
\section{\texorpdfstring{\textcolor{Blue}{\textsc{Part III: Models of Digitalization}}}{}}\label{section-2}}

\vspace{0.5cm}

\hypertarget{week-9-may-6-the-united-states-all-hail-the-market}{%
\subsection{Week 9, May 6 : The United States: All hail the
Market?}\label{week-9-may-6-the-united-states-all-hail-the-market}}

Farrell, H., \& Newman, A. L. (2021). The Janus Face of the Liberal
International Information Order: When Global Institutions Are
Self-Undermining. International Organization, pp.~1--26.

Newman, A. L., \& Bach, D. (2004). Self-Regulatory Trajectories in the
Shadow of Public Power. Resolving Digital Dilemmas in Europe and the
United States. Governance, 17(3), pp.~387--413.

\hypertarget{week-10-may-13}{%
\subsection{Week 10, May 13 :}\label{week-10-may-13}}

\emph{We do not have class this week due to public holidays}

\hypertarget{week-11-may-20-china-state-surveillance-capitalism}{%
\subsection{Week 11, May 20 : China: State Surveillance
Capitalism?}\label{week-11-may-20-china-state-surveillance-capitalism}}

Jia, L. and Winseck, D. (2018) `The political economy of Chinese
internet companies: Financialization, concentration, and
capitalization', International Communication Gazette, 80(1),
pp.~30-\/--59.

Shen, H. (2018) `Building a Digital Silk Road? Situating the Internet in
China's Belt and Road Initiative', International Journal of
Communication, 12, pp.~2683---2701.

\hypertarget{week-12-may-27-the-european-union-a-third-way}{%
\subsection{Week 12, May 27 : The European Union: A Third
Way?}\label{week-12-may-27-the-european-union-a-third-way}}

Floridi, L. (2020). The Fight for Digital Sovereignty: What It Is, and
Why It Matters, Especially for the EU. Philosophy \& Technology, 33(3),
pp.~369--378.

Radu, R., \& Chenou, J.-M. (2015). Data control and digital regulatory
space (s): Towards a new European approach. Internet Policy Review,
Journal on Internet Regulation, 4(2).

\hypertarget{section-3}{%
\section{\texorpdfstring{\textcolor{Blue}{\textsc{Part IV: The EU's Digital Agenda}}}{}}\label{section-3}}

\vspace{0.5cm}

\hypertarget{week-13-june-3}{%
\subsection{Week 13, June 3 :}\label{week-13-june-3}}

\emph{We do not have class this week due to public holidays}

\hypertarget{week-14-june-10-data-protection}{%
\subsection{Week 14, June 10 : Data
Protection}\label{week-14-june-10-data-protection}}

\emph{We do not have class this week due to public holidays}

Rossi, A. (2018) `How the Snowden Revelations Saved the EU General Data
Protection Regulation', The International Spectator, 53(4), pp.~95-111.

Farrell, H., \& Newman, A. L. (2018). Linkage Politics and Complex
Governance in Transatlantic Surveillance. World Politics, 70(4),
515--554.

\hypertarget{week-15-june-17-taxation}{%
\subsection{Week 15, June 17 :
Taxation}\label{week-15-june-17-taxation}}

Danescu, E. (2020). Taxing intangible assets: Issues and challenges for
a digital Europe. Internet Histories, 4(2), 196--216.

Lips, W. (2020). The EU Commission's digital tax proposals and its
cross-platform impact in the EU and the OECD. Journal of European
Integration, 42(7), 975--990.

\hypertarget{week-16-june-24-policy-session-digital-services-actdigital-markets-act}{%
\subsection{Week 16, June 24 : Policy Session: Digital Services
Act/Digital Markets
Act}\label{week-16-june-24-policy-session-digital-services-actdigital-markets-act}}

tba




\end{document}

\makeatletter
\def\@maketitle{%
  \newpage
%  \null
%  \vskip 2em%
%  \begin{center}%
  \let \footnote \thanks
    {\fontsize{18}{20}\selectfont\raggedright  \setlength{\parindent}{0pt} \@title \par}%
}
%\fi
\makeatother